\documentclass[a4paper,11pt]{article}

\usepackage[ngerman]{babel}
\usepackage[utf8]{inputenc}

\begin{document}

\section*{Wahrheitsgemäße Erklärung}
\large
Ich erkläre hiermit wahrheitsgemäß, dass ich
\begin{itemize}
\item die eingereichte Arbeit selbständig und ohne unerlaubte Hilfe angefertigt habe,
\item außer den im Schrifttumsverzeichnis angegebenen Hilfsmitteln keine weiteren
benutzt und alle Stellen, die aus dem Schrifttum ganz oder annähernd
entnommen sind, als solche kenntlich gemacht und einzeln nach ihrer Herkunft
unter Bezeichnung der Ausgabe (Auflage und Jahr des Erscheinens), des Bandes
und der Seite des benutzten Werkes, bei Internet-Quellen unter Angabe der
vollständigen Adresse und des Sichtungsdatums, in der Abhandlung
nachgewiesen habe,
\item alle Stellen und Personen, welche mich bei der Vorbereitung und Anfertigung der
Abhandlung unterstützten, genannt habe,
\item die Abhandlung noch keiner anderen Stelle zur Prüfung vorgelegt habe und
dieselbe noch nicht anderen Zwecken -- auch nicht teilweise -- gedient hat.
\end{itemize}

\noindent
Mir ist bewusst, dass jeder Verstoß gegen diese Erklärung eine Bewertung der
eingereichten Arbeit mit Note \glqq nicht bestanden\grqq\ zur Folge hat.

\vspace{8mm}

\hspace*{\fill}\begin{tabular}{@{}l@{}}\hline
\makebox[8cm]{Ort, Datum \quad Markus Opolka}
\end{tabular}

\end{document}
